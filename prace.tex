% ============================================================================ %
%
%           Šablona bakalářské/diplomové práce
%
% Autor:    Ing. Jozef Říha (4. květen 2006)
%           (některé komentáře převzaty z dokumentu Ivana Pomykacze)
%
% Verze:    2017-01-19, Ing. Pavel Tomášek (tomasek@fai.utb.cz)
%
% Kódování: UTF-8 (žluťoučký kůň úpěl ďábelšké ódy)
%
% Sazba:    pdflatex prace.tex && pdflatex prace.tex
%           (nutné dvakrát pro korektní vložení citací a jiných referencí),
%           v případě umístění literatury do externího bib souboru je třeba volat
%           pdflatex statement.tex && bibtex statement.tex && pdflatex statement.tex && pdflatex statement.tex
%
% Tip:      Ve správně vysázeném českém textu by na konci řádku neměla zůstant
%           samotná jednopísmenná předložka. Na takové místo se vkládá
%           nezalomitelná mezera pomocí symbolu ~. Existuje program, který umí
%           zpracovat celý TeX dokument najednou podle českých konvencí:
%           http://petr.olsak.net/ftp/olsak/vlna/
%
%           Pozor! Vzhledem k požadovanému standardu PDF/A nesmí obrázky obsahovat 
%           alfa kanál (průhlednost).
%
% ============================================================================ %


\documentclass[a4paper,12pt]{article}

% Definice vzhledu a nastavení se načítá z následujícího souboru (netřeba editovat)
\input{tex/UTB.tex}

% Uživatelské definice -- upravte dle požadavků
\nastavfakultu{FAI}
	% FAI  -- pro Fakultu aplikované informatiky
	% FAME -- pro Fakultu managementu a ekonomiky
	% FHS  -- pro Fakultu humanitních studií
	% FLKR -- pro Fakultu logistiky a krizového řízení
	% FMK  -- pro Fakutlu mutimediálních komunikací
	% FT   -- pro Fakultu technologickou
	% UNI  -- pro Univerzitní institut
\nastavtyp{SP}
	% BP   -- bakalářská práce
	% DP   -- diplomová práce
	% SP	   -- semestrální práce
\nastavrok{2022}
	% zadejte rok místo "xxxx"
\nastavjazyk{CZ}
	% CZ   -- práce bude v českém jazyce
	% EN   -- práce bude v anglickém jazyce

% Lze přidat vertikalni odsazeni nad (prvni parametr) a pod (druhy parametr)
% obrázky, tabulky i rovnice/soustavy rovnic
\nastavmezerukolemobrazku{0mm}{0mm}
\nastavmezerukolemtabulek{0mm}{0mm}
\nastavmezerukolemrovnic{0mm}{0mm}

\nastavautora{Bc. Petr Vykoukal}
\nastavnazevcz{Vybraný ransomware za období 2002-2022}
\nastavnazeven{Název práce anglicky (max. 2 řádky)} % Jen u anglicky psané práce
\nastavabstraktcz{Semestrální práce se zabývá vývojem ransomwaru od roku 2002 do současnosti a dále se věnuje několika vybraným ransomwarům, které se v tomto období vyskytly.}
\nastavabstrakten{The semester's thesis deals with the development of Ransomware from 2002 to the present day and also focuses on several selected Ransomware that occurred during this period.}
\nastavklicovaslovacz{Malware, Ransomware}
\nastavklicovaslovaen{Malware, Ransomware}

% Následující příkaz nastaví standard PDF/A-1b
\aplikujpdfa


%\usepackage{graphicx}
\usepackage{adjustbox}
\graphicspath{{./graphics/}}

\usepackage{booktabs}
\usepackage{array}

\usepackage{xurl}


%\hyphenation{u-kon-čo-val} - přidá do slovníku slovo ukončoval s možnostmi, jak ho rozdělovat. Odtud už stačí jen kdykoliv použít slovo "ukončoval"
%dalši varianta je použít slovo u\-kon\-čo\-val
\hyphenation{u-kon-čo-val na-pa-dal na-pa-de-né sy-sté-mem o-prá-vně-ní ran-som-ware
	to-má-še sou-bo-ry u-pra-vil od-stra-nit i-den-ti-fi-ko-vat bli-ka-jí-cím 
	in-sta-lo-vat způ-so-be-né al-go-rit-mem nej-čas-tě-ji ne-za-pla-ti-la
	ši-fro-va-cí pos-ky-tnu-té-ho při-chá-ze-ly žád-ná va-ri-an-ty web-ka-me-ra
	zra-ni-tel-nos-ti za-ši-fro-vá-ní sou-bo-ru vy-ge-ne-ro-val}  

% ============================================================================ %
\begin{document}

\voffset=\valueVOffset\evensidemargin=\valueSideMargin\oddsidemargin=\valueSideMargin\headsep=\valueHeadSep\headheight=\valueHeadHeight\setlength{\parskip}{3pt}\textheight=\valueTextHeight\textwidth=\valueTextWidth

\titulnistrana
\clearpage

%\zadani
%následující řádek je místo zadání (nascanováná strana blablabla...)
\voffset=\valueVOffset\evensidemargin=\valueSideMargin\oddsidemargin=\valueSideMargin\headsep=\valueHeadSep\headheight=\valueHeadHeight\setlength{\parskip}{3pt}\textheight=\valueTextHeight\textwidth=\valueTextWidth

%\prohlaseni

\abstraktaklicovaslova


% ============================================================================ %
\clearpage
\thispagestyle{empty}
%Zde je místo pro případné poděkování, motto, úryvky knih, básní atp.


% ============================================================================ %
\obsah  % Obsah je generován automaticky


% ============================================================================ %
\OdsazovaniOdstavcuStart % Nastaví odsazování odstavců dle zvoleného jazyka

% ============================================================================ %
% Encoding: UTF-8 (žluťoučký kůň úpěl ďábelšké ódy)
% ============================================================================ %

% ============================================================================ %
\nn{Úvod}
V této práci se zabývám vývojem ransomwaru v~období 2002-2022, konkrétně tedy deseti vybranými. ransomware je konkrétní druh škodlivého software, který je specifický tím, že uživatele napadených počítačů nějakou formou vydírá za~účelem vymámení platby -- výkupného. \uv{Rukojmí} v~tomto procesu mohou být různá. Může se jednat o~zašifrovaná data, nebo výhružku, že~citlivá data uživatele budou zveřejněna na Internetu, nebo přímo kontaktům oběti. V~následující kapitole se~tedy postupně budu zabývat touto tématikou v~chronologickém pořadí, jak jednotlivé hrozby přicházely a nakonec v kapitole č. \ref{fig:gantt} uvedu Ganttův diagram, který bude zobrazovat časové působení jednotlivých ransomwarů.



% ============================================================================ %
%\cast{Teoretická část}

\n{1}{Vybraný ransomware}
V~této kapitole postupně uvedu jednotlivé ransomwary včetně jejich popisu. Seznam je chronologicky seřazen podle toho, jak jednotlivé hrozby přicházely.

\n{2}{GPcode}
Ransomware GPcode napadal operační systém Windows. Na scéně se objevil v červnu 2006 \cite{gpcode-knowbe4}. Šířil se pomocí emailových příloh. Konkrétně se jednalo o skodlivý dokument aplikace Word, který předstíral, že jde o žádost o zaměstnání. Při jeho spuštění upravil registr systému Windows tak, aby došlo ke spuštění GPcode při spuštění počítače. Následně začal vyhledávat na lokálních i vzdálených discích soubory určitého typu\footnote{Uváděny \cite{gpcode-fsecure} jsou soubory typu: \texttt{xls}, \texttt{doc}, \texttt{txt}, \texttt{rtf}, \texttt{zip}, \texttt{rar}, \texttt{dbf}, \texttt{htm}, \texttt{html}, \texttt{jpg}, \texttt{db}, \texttt{db1}, \texttt{db2}, \texttt{asc}, \texttt{pgp}.  V jiném zdroji \cite{gpcode-riskfront} je ale uvedeno, že se může jednat až o 80 různých druhů souborů.}.


\begin{figure}
	\begin{center}
		\includegraphics[scale=0.6]{clocker1}
		\caption{Náhled žádosti o výkupné CryptoLocker, převzato z \url{https://www.avast.com/c-cryptolocker}}
		\label{fig:clocker1}
	\end{center}
\end{figure}


\begin{table}[h]
	\begin{center}
		\caption{Nejvyšší způsobené škody způsobené ransomwarem \texttt{notPetya}}
		\vspace{0.3cm}
		\begin{tabular}{m{22em}m{8em}}
			\toprule
			Farmaceutická společnost Merck \cite{merck} & 40\,000 počítačů a 1\,400\,000\,000\,USD \\
			\midrule
			Francouzská stavební společnost Saint--Gobain \cite{sweek} & 387\,000\,000\,USD \\
			\bottomrule
		\end{tabular}
		\label{notpetya-impacts}
	\end{center}
\end{table}



\n{1}{Ganttův Diagram}

%\begin{figure}[h!]
%	\begin{center}
%		\includegraphics[scale=0.17, angle=90]{gant}
%		\caption{Ganttův diagram}
%		\label{fig:gantt}
%	\end{center}
%\end{figure}


\begin{figure}[h!]
	\begin{adjustbox}
    {addcode={
        \begin{minipage}{\width}}{\caption{Ganttův diagram}
        \label{fig:gantt}
        \end{minipage}},rotate=90,center}
%        %\includegraphics[trim=left bottom right top, clip]{file}
      \includegraphics[scale=0.18]{gant}
  \end{adjustbox}
\end{figure}


% ============================================================================ %
% Pokud Vaše práce neobsahuje analytickou část, stačí odstranit či zakomentovat nasledujících pár rádků
%\cast{Analytická část}
%\n{1}{Nadpis}
%\n{2}{Podnadpis}
% ============================================================================ %
%\cast{Projektová část}
%\n{1}{Nadpis}
%\n{2}{Podnadpis}
% ============================================================================ %
\nn{Závěr}
Na základě uvedených příkladů je zřejmé, že ransomware prošel od svého počátku velkým rozvojem. Z velké části se šíří pomocí emailových zpráv, kdy je přiložen samotný ransomware, nebo program, případně skript, který jej stáhne a naintaluje do uživatelova počítače. Úživatelé tak mohli zabránit infekci svého počítače pouze větší obezřejností. Na rozdíl od útoků, které zneužily tzv. zero-day zranitelnosti.

Dále po útoku ransomwaru \texttt{Reveton}, o kterém jsem psal v kapitole č.~\ref{reveton}, je znatelná změna ve vývoji ransomwaru, který používali sami vývojáři do modelu, kdy je ransomware poskytován za úplatu jako služba -- RaaS.


% ============================================================================ %


\OdsazovaniOdstavcuStop


% ============================================================================ %
\seznamlit{
  % Na toto místo vložit veškeré citované bibliografické položky.
  % http://generator-citaci.cz/
	\bibitem{flashpoint-history}
	{
		The History and Evolution of Ransomware Attacks | Flashpoint.
		\it{Flashpoint | Cyber Threat Intelligence Platform \& Professional Services} [online].
		Copyright © 2022 Flashpoint. All rights reserved. [cit. 2022-10-13].
		Dostupný z~WWW: \url{https://flashpoint.io/blog/the-history-and-evolution-of-ransomware-attacks/}
	}  
	
}

% Pro generování literatury lze alternativně použít i příkaz "\seznamlitbib", 
% který se postará o plnohodnotné vkládání referencí pomocí "bibliography". 
% V takovém případě se využívají bibliografické údaje uložené v souboru 
% tex-literatura.bib. Ty se automaticky upravuji dle zvolené citační normy 
% (v šabloně je nastavena korektní česká norma).
%\seznamlitbib


% ============================================================================ %
% ============================================================================ %
% Encoding: UTF-8 (žluťoučký kůň úpěl ďábelšké ódy)
% ============================================================================ %

\seznamzkr

\begin{tabular}{ll}
	SMB		& Server Message Block\\	
	FTP		& File Transfer Protocol\\
	ASCII	& American Standard Code for Information Interchange\\
	SMTP		& Simple Mail Transfer Protocol\\

\end{tabular}

% ============================================================================ %
 % Seznam zkratek


% ============================================================================ %
\seznamobr  % Seznam je generován automaticky


% ============================================================================ %
\seznamtab  % Seznam je generován automaticky


% ============================================================================ %
%\input{tex/prilohy.tex} % Prilohy


% ============================================================================ %

\end{document}

% ============================================================================ %
