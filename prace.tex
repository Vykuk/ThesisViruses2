% ============================================================================ %
%
%           Šablona bakalářské/diplomové práce
%
% Autor:    Ing. Jozef Říha (4. květen 2006)
%           (některé komentáře převzaty z dokumentu Ivana Pomykacze)
%
% Verze:    2017-01-19, Ing. Pavel Tomášek (tomasek@fai.utb.cz)
%
% Kódování: UTF-8 (žluťoučký kůň úpěl ďábelšké ódy)
%
% Sazba:    pdflatex prace.tex && pdflatex prace.tex
%           (nutné dvakrát pro korektní vložení citací a jiných referencí),
%           v případě umístění literatury do externího bib souboru je třeba volat
%           pdflatex statement.tex && bibtex statement.tex && pdflatex statement.tex && pdflatex statement.tex
%
% Tip:      Ve správně vysázeném českém textu by na konci řádku neměla zůstant
%           samotná jednopísmenná předložka. Na takové místo se vkládá
%           nezalomitelná mezera pomocí symbolu ~. Existuje program, který umí
%           zpracovat celý TeX dokument najednou podle českých konvencí:
%           http://petr.olsak.net/ftp/olsak/vlna/
%
%           Pozor! Vzhledem k požadovanému standardu PDF/A nesmí obrázky obsahovat 
%           alfa kanál (průhlednost).
%
% ============================================================================ %


\documentclass[a4paper,12pt]{article}

% Definice vzhledu a nastavení se načítá z následujícího souboru (netřeba editovat)
\input{tex/UTB.tex}

% Uživatelské definice -- upravte dle požadavků
\nastavfakultu{FAI}
	% FAI  -- pro Fakultu aplikované informatiky
	% FAME -- pro Fakultu managementu a ekonomiky
	% FHS  -- pro Fakultu humanitních studií
	% FLKR -- pro Fakultu logistiky a krizového řízení
	% FMK  -- pro Fakutlu mutimediálních komunikací
	% FT   -- pro Fakultu technologickou
	% UNI  -- pro Univerzitní institut
\nastavtyp{SP}
	% BP   -- bakalářská práce
	% DP   -- diplomová práce
	% SP	   -- semestrální práce
\nastavrok{2022}
	% zadejte rok místo "xxxx"
\nastavjazyk{CZ}
	% CZ   -- práce bude v českém jazyce
	% EN   -- práce bude v anglickém jazyce

% Lze přidat vertikalni odsazeni nad (prvni parametr) a pod (druhy parametr)
% obrázky, tabulky i rovnice/soustavy rovnic
\nastavmezerukolemobrazku{0mm}{0mm}
\nastavmezerukolemtabulek{0mm}{0mm}
\nastavmezerukolemrovnic{0mm}{0mm}

\nastavautora{Bc. Petr Vykoukal}
\nastavnazevcz{Vývoj počítačových červů 2005-2022}
\nastavnazeven{Název práce anglicky (max. 2 řádky)} % Jen u anglicky psané práce
\nastavabstraktcz{Semestrální práce se zabývá vývojem počítačových od roku 2005 do současnosti a dále se věnuje několika vybraným červům, které se v tomto období vyskytli.}
\nastavabstrakten{The semester's thesis deals with the development of Computer Worms from 2005 to the present day and also focuses on several selected Worms that occurred during this period.}
\nastavklicovaslovacz{Malware, Worm}
\nastavklicovaslovaen{Malware, Worm}

% Následující příkaz nastaví standard PDF/A-1b
\aplikujpdfa


%\usepackage{graphicx}
\usepackage{adjustbox}
\graphicspath{{./graphics/}}

\usepackage{booktabs}
\usepackage{array}

\usepackage{xurl}


%\hyphenation{u-kon-čo-val} - přidá do slovníku slovo ukončoval s možnostmi, jak ho rozdělovat. Odtud už stačí jen kdykoliv použít slovo "ukončoval"
%dalši varianta je použít slovo u\-kon\-čo\-val
\hyphenation{u-kon-čo-val na-pa-dal na-pa-de-né sy-sté-mem o-prá-vně-ní ran-som-ware
	to-má-še sou-bo-ry u-pra-vil od-stra-nit i-den-ti-fi-ko-vat bli-ka-jí-cím 
	in-sta-lo-vat způ-so-be-né al-go-rit-mem nej-čas-tě-ji ne-za-pla-ti-la
	ši-fro-va-cí pos-ky-tnu-té-ho při-chá-ze-ly žád-ná va-ri-an-ty web-ka-me-ra
	zra-ni-tel-nos-ti za-ši-fro-vá-ní sou-bo-ru vy-ge-ne-ro-val za-hr-nu-jí-cí
	na-pa-de-né-mu}  

% ============================================================================ %
\begin{document}

\voffset=\valueVOffset\evensidemargin=\valueSideMargin\oddsidemargin=\valueSideMargin\headsep=\valueHeadSep\headheight=\valueHeadHeight\setlength{\parskip}{3pt}\textheight=\valueTextHeight\textwidth=\valueTextWidth

\titulnistrana
\clearpage

%\zadani
%následující řádek je místo zadání (nascanováná strana blablabla...)
\voffset=\valueVOffset\evensidemargin=\valueSideMargin\oddsidemargin=\valueSideMargin\headsep=\valueHeadSep\headheight=\valueHeadHeight\setlength{\parskip}{3pt}\textheight=\valueTextHeight\textwidth=\valueTextWidth

%\prohlaseni

\abstraktaklicovaslova


% ============================================================================ %
\clearpage
\thispagestyle{empty}
%Zde je místo pro případné poděkování, motto, úryvky knih, básní atp.


% ============================================================================ %
\obsah  % Obsah je generován automaticky


% ============================================================================ %
\OdsazovaniOdstavcuStart % Nastaví odsazování odstavců dle zvoleného jazyka

% ============================================================================ %
% Encoding: UTF-8 (žluťoučký kůň úpěl ďábelšké ódy)
% ============================================================================ %

% ============================================================================ %
\nn{Úvod}
\cite{wiki-timeline}
\textcolor{red}{V této práci se zabývám vývojem ransomwaru v~období 2002-2022, konkrétně tedy deseti vybranými. ransomware je konkrétní druh škodlivého software, který je specifický tím, že uživatele napadených počítačů nějakou formou vydírá za~účelem vymámení platby -- výkupného. \uv{Rukojmí} v~tomto procesu mohou být různá. Může se jednat o~zašifrovaná data, nebo výhružku, že~citlivá data uživatele budou zveřejněna na Internetu, nebo přímo kontaktům oběti. V~následující kapitole se~tedy postupně budu zabývat touto tématikou v~chronologickém pořadí, jak jednotlivé hrozby přicházely a nakonec v kapitole č. \ref{fig:gantt} uvedu Ganttův diagram, který bude zobrazovat časové působení jednotlivých ransomwarů.}



% ============================================================================ %
%\cast{Teoretická část}

\n{1}{Zotob}
První mnou vybraný červ jménem \texttt{Zotob} se objevil 14. srpna 2005\cite{wiki-timeline}\cite{zotob-schneider}. Cílem tohoto červa byly počítače s operačním systémem Windows 2000\cite{zotob-fsecure}.

Po spuštění červa v napadeném počítači uložil svou kopii s názvem \texttt{botzor.exe} do systémové složky\cite{zotob-fsecure}. Dále vytvořil záznam v registrech systému, aby umožnil spuštění programu po startu operačního systému. Dále vytvořil mutex s názvem \texttt{B-O-T-Z-O-R}, aby zabráníl spuštění spuštění více, než jedné instance programu. V dalších verzích tohoto červa byl postup stejný, jen se soubory jmenovaly \texttt{csm.exe} ve verzi \texttt{Zotob.B} a \texttt{per.exe} ve verzi \texttt{Zotob.C}.

Název použitého mutexu odpovídá názvu, kterým červa pojmenoval jeho autor. Tedy \texttt{Batzor2005}. Název \texttt{Zotob} pochází od antivirových společností, které hrozbu detekovaly\cite{zotob-wikidot}. Současně se skutečný název červa ve verzi \texttt{Zotob.A} a \texttt{Zotob.B} zapsán do souboru \texttt{hosts}. Konkrétně je tam tedy vepsána zpráva \uv{Botzor2005 Made By .... Greetz to good friend Coder. Based On HellBot3} a \uv{MSG to avs: the first av who detect this worm will be the first killed in the next 24hours!!!}\cite{zotob-fsecure}. Soubor \texttt{hosts} je také upraven, aby přesměroval přístup k některým serverům\footnote{Kompletní seznam je k vidění na stránce \url{https://www.f-secure.com/v-descs/zotob_a.shtml}} na adresu \texttt{127.0.0.1}, tedy \texttt{localhost}\cite{zotob-fsecure}. Také došlo ke spuštění FTP serveru na portu 33333, odkud si mohly další napadené systémy stáhnout tohoto červa\cite{zotob-fsecure}.

Ke svému šíření využíval tento červ ve všech verzích zranitelnost MS Windows Plug and Play služby označenou jako \texttt{MS05-039}\cite{zotob-fsecure}. Ve verzi \texttt{Zotob.A} vytvořil 300 vláken, ve verzcíh \texttt{Zotob.B} a \texttt{Zotob.C} pak 200 vláken, která zkoušely vytvořit spojení k náhodným IP adresám generovaným z IP adresy napadeného počítače podle masky podsítě 255.255.0.0\cite{zotob-fsecure}\cite{zotobb-fsecure}\cite{zotobc-fsecure}. Na nalezených IP adresách se pokusil navázal TCP spojení na portu 445 a využít výše uvedené zranitelnosti.

Jedná se ozranitelnost, která umožňuje vzálené spuštění kódu. Touto zranitelností trpělo více operačních systémů, ale právě u Windows 2000 bylo možné vzdálené spuštění kódu i bez ověření uživatele, zatím co na jiných systémech bylo nutné přihlášení\cite{zotob-ms}\cite{zotob-msbulletin}.

Pokud bylo úspěšně navázáno spojení s cílovým systémem protokolem TCP na portu 445, kde běží služba \texttt{Microsoft-ds} \texttt{SMB}\footnote{SMB -- Server Message Block} a podařilo se vzdálené spuštění kódu, došlo ke spuštění shellu na cílovém počítači, který naslouchal na portu 8888. Na tento port byl zaslán FTP sctipt, který zařídil stažení červa skrz FTP do napadeného počítače, kde běží FTP server na portu 33333, jako souboru \texttt{haha.exe} a jeho spuštění\cite{zotob-fsecure}\cite{zotobb-fsecure}.

Každý napadený počítač dokázal přijímat příkazy pomocí IRC a předdefinovaných adresách. Útočník tak mohl dát příkazy například k vyžádání informací o systému, stažení spustitelného souboru, manipulaci s bezpečnostním nastavením systému, nebo odstranení červa ze systému.

Verze \texttt{Zotob.C} pak dokázala využít zranitelnosti v Abstract Syntax Notation 1 popsané jako \texttt{MS04-007}, která učinila zranitelným i systém Windows XP do SP1\cite{zotob-msbulletin2}. Také tato verze využívala ke svému šíření metody phishingu pomocí emailových zpráv.\cite{zotobc-fsecure}

25. srpna 2005 byli zatčeni tři lidé zodpovědní za vznik tohoto červa. Jeho autorem byl osumnáctiletý Maročan, narozený v Rusku, Farid Essebar, který vystupoval pod přezdívkou \uv{Diab10} a jeho dvacetiletý přítel Achraf Bahloul. V Turecku byl zatčen také 21 letý Artilla Ekici s přezdívkou \uv{Coder}, který Essebarovi za vytvoření červa zaplatil.\cite{zotob-schneider}\cite{zotob-fbi}

Zasaženo bylo více než 100 amerických spočeností\cite{zotob-fbi}. Průměrná cena za zotavení z infekce červem \texttt{Zotob} je podle průzkumu společnosti Cybertrust 97\,000\,USD. 61 procent obětí uvedlo, že vyčištění vyžadovalo 80 hodin práce.\cite{zotob-wikidot}\cite{zotob-cnet} Původní zpráva společnosti však již není dostupná.

\n{1}{Nyxem.E}
Další z vybraných červů se objevil 20. ledna 2006\cite{nyxem-wiki}. Jeho cílem byl operační systém Windows. Je také znám pod dalšími jmény, jako \texttt{Blackworm}, \texttt{Kama Sutra}, \texttt{MyWife}, \texttt{CME-24}\dots\cite{nyxem-caida} Je napsán ve Visual Basicu a skompilován jako spustitelný soubor p-code a jeho velikost je 95kB. Jeho tělo obsahuje řetězce \uv{mysoulmystfly}, \uv{offlinehacker}, uv{evilpain} a \uv{setthesun}.\cite{nyxem-fsecure} 

Dokázal se šířit elektronickou poštou a nebo pomocí sdílených jednotek v síti. Napadenému uživateli tak mohl přijít jako příloha v emailu, kdy se snažil tvářit jako archiv \texttt{ZIP}. Pokud uživatel tento soubor spustil, otevřel se uživateli program WinZip, aby nabyl dojmu, že otevřel obyčejný archiv\cite{nyxem-trendmicro}. Současně také mohlo dojít k zablokování vstupů z klávesnice a myši, čím mělo být zabráněno uživateli v odhlášení, nebo spuštění správce úloh\cite{nyxem-fsecure}\cite{nyxem-trendmicro}. Také je kontaktována určitá webová stránka, která slouží jako počítadlo populace infikovaných počítačů.

V této chvíli červ vytvoří své kopie ve složce \texttt{\%WINDOWS\%}\footnote{Obvykle se 
jedná o složku C:\textbackslash Windows\textbackslash } soubor \texttt{rundll16.exe}, ve složce \texttt{\%SYSTEM\%}\footnote{Obvykle se jedná o složku C:\textbackslash Windows\textbackslash System32\textbackslash } soubory \texttt{scanregw.exe}, \texttt{Update.exe}, \texttt{Winzip.exe}, \texttt{WINZIP\_TMP.exe} a ve složce \texttt{\%SYSTEM ROOT\%\textbackslash Documents and Settings\textbackslash All Users\textbackslash Start Menu\textbackslash Pro\-grams\textbackslash Startup\textbackslash }\footnote{\%SYSTEEM ROOT\% je nahrazen kořenovým adresářem, kde je instalován Windows, tedy obvykle C:\textbackslash } soubor \texttt{Quick Pick.exe}.\cite{nyxem-fsecure}\cite{nyxem-trendmicro}

Následně červ upraví záznamy v registrech systému, aby v průzkumníku souborů zakázal zobrazení skrytých souborů, zajistil si spuštění při startu počítače a zabránil při startu spuštění různých antivirových produktů. Dále prohledá složky, kde se obvykle nachází antivirové programy, a pokusí se vymazat některé, případně všechny, jejich soubory. Mimo to se snaží odstanit i jiné programy, jako například \texttt{LimeWire}\footnote{LimeWire - je program určený pro P2P sdílení}.\cite{nyxem-caida}\cite{nyxem-fsecure}\cite{nyxem-trendmicro}

Ve chvíli, kdy je počítač infikován, tak kontroluje spouštěné programy a pokud nalezne titulek obsahující konkrétní řetězec, tak se pokusí program ukončit\cite{nyxem-fsecure}. Přiklad je k vidění i na YouTube\cite{nyxem-youtube}, kde v jedné části dochází k ukončení aplikace \texttt{Notepad}, které otevírá soubor \texttt{SCAN.txt}, čímž se jí do titulku dostalo hledané klíčové slovo.

Jednou měsíčně, konkrétně každého 3. dne v měsíci, počínaje dnem 3. 2. 2006, je pak spuštěn program \texttt{Update.exe}, který v počítači vyhledá všechny soubory s koncovkami \texttt{DOC}, \texttt{XLS}, \texttt{MDB}, \texttt{MDE}, \texttt{PPT}, \texttt{PPS}, \texttt{ZIP}, \texttt{RAR}, \texttt{PDF}, \texttt{PSD} a \texttt{DMP} a obsah těchto souborů přešpíše 32B dlouhým řetězcem, který převeden do ASCII zní \uv{DATA ERROR [47 0F 94 93 F4 K5]}. Poslední 2B jsou hodnoty \texttt{0D 0A}, což jsou netisknutelné znaky pro konec řádku.\cite{nyxem-securelist}\cite{nyxem-fsecure}

V počítači a ve složce, kde je uložena cache programu Internet Explorer vyhledává soubory s určitými koncovkami a snaží se z nich vytěžit emailové adresy. Pokud tyto adresy ale obsahují určitá klíčová slova, jako například \uv{mcafee}, nebo \uv{spam}, tak je ignoruje\cite{nyxem-fsecure}. Červ obsahuje svůj vlastní SMTP engine\cite{nyxem-trendmicro} a pomocí něj se snaží šířit na vytěžené emailové adresy. Má své předdefinované předměty a těla zpráv. Sám sebe pak do zpráv připojí jako přílohu.\cite{nyxem-fsecure}

Pro šíření v síti má dvě rutiny. První z nich prohlídne v registru umístění složek, kde jsou osobní dokumenty, nebo nedávno otevřené dokumenty. Pokud takové nalezne, tak složky projde a náhodně si vybere název souboru a připojí k němu koncovku \texttt{EXE}. Následně vyhledá sdílené jednotky v síti a na vyhledané jednotce vytvoří nový soubor se vzniklým jménem, který ale obsahuje tělo červa. Pokud nenalezne žádný vhondý soubor, jehož název by mohl okopírovat, tak použije \uv{New WinZip File.exe}, \uv{Zipped Files.exe}, nebo \uv{movies.exe}.\cite{nyxem-fsecure}

Druhá varianta je pak, že prohledá sdílené jednotky v síti a tam se snaží zkopírovat jako:
\begin{itemize}
	\item \textbackslash Admin\$\textbackslash WINZIP\_TMP.exe
	\item \textbackslash c\$\textbackslash WINZIP\_TMP.exe
	\item \textbackslash c\$\textbackslash Documents and Settings\textbackslash All Users\textbackslash Start Menu\textbackslash Programs\textbackslash Startup\textbackslash WinZip Quick Pick.exe
	\begin{itemize}
		\item V tomto případě ve stejné složce odstraní soubor \uv{WinZip Quick Pick.lnk}
	\end{itemize}
\end{itemize}
Také se snaží zjistit, zda existuje na sdílené jednotce antivirový program tím, že prohledává konkrétná složky jako napříkald \uv{\textbackslash c\$\textbackslash Program Files\textbackslash Norton AntiVirus}, a pokud nalezne, tak se je snaží odstanit. Také se pokusí na vzáleném počítači spustit plánovanou úlohu na 59. minutu aktuální hodiny, která by spustila nakopírované tělo červa.\cite{nyxem-fsecure}

Tím, že \texttt{Nyxem} při své instalaci kontaktuje určitou webovou stránku, která funguje jako počítadlo populace, je mělo by být možné odhadnout kolik počítačů bylo infikováno. Dle analýzy Davida Moora a Collen Shannon je odhad infikovaných počítačů mezi 469\,507 až 946\,835 ve více než 200 zemích za obcodbí 15. ledna 2006 23:40:45\,UTC do 1. února 2006 05:00:12\,UTC.\cite{nyxem-caida}

Přes to však dle zprávy společnosti F-Secure může být započítáno více přístupů z jedné IP adresy. Výsledné číslo podle nich pak vychází na více než 300\,000 unikátních IP adres.\cite{nyxem-fsecure}


\n{1}{Brontok}
varianta N koncem března 2006 - mass-email worm s původem v indonesii
\n{2}{Cíl útoku}
\n{2}{Předpokládaný počet napadených zařízení}
\n{2}{Identifikační znaky}
\n{2}{Způsob šíření}

\n{1}{Stration/Warezov}
konec září 2006
\n{2}{Cíl útoku}
\n{2}{Předpokládaný počet napadených zařízení}
\n{2}{Identifikační znaky}
\n{2}{Způsob šíření}

\n{1}{Storm Worm}
17. ledna 2007 Storm Worm - fast spreading email spamming thread to MS systems. Zařazuje infikované počítače to \"Storm botnet\". Kolem 30. června 2007 infikoval 1.7 milionů počítačů a do září napadl mezi 1 a 10 miliony počítačů. Předpokládá se, že pochází z Ruska. Maskuje se jako zpravodajský email obsahující film o falešných novinářských příbězích a žádá o stažení přílohy o které tvrdí, že je film.
\n{2}{Cíl útoku}
\n{2}{Předpokládaný počet napadených zařízení}
\n{2}{Identifikační znaky}
\n{2}{Způsob šíření}

\n{1}{Koobface}
červenec 2008 - cílí na uživatele facebooku a myspace
\n{2}{Cíl útoku}
\n{2}{Předpokládaný počet napadených zařízení}
\n{2}{Identifikační znaky}
\n{2}{Způsob šíření}

\n{1}{Conficker}
listopadu 2008 - infikoval 9 až 15 milionů MS server systémů od Windows 2000 do Windows 7 beta. Microsoft vypsal odměnu 250tis USD za informaci vedoucí k dopadení autora. Bylo 5 variant A-(21.list.08) B-(29.pros.08) C-(20.únor.09) D-(4.břez.09) a E-(7.dub.09). 16. prosince 2008 vydal MS záplatu KB958644 opravující zranitelnost odpovědnou za šíření confickera.
\n{2}{Cíl útoku}
\n{2}{Předpokládaný počet napadených zařízení}
\n{2}{Identifikační znaky}
\n{2}{Způsob šíření}

\n{1}{Daprosy Worm}
15. července 2009 (objevil Symantec) - trojan worm, vytvoření ke krádeži hesel do online her v internetových kavárnách. Mohl zachytit všechny stisknuté klávesy a zaslat autorovi.
\n{2}{Cíl útoku}
\n{2}{Předpokládaný počet napadených zařízení}
\n{2}{Identifikační znaky}
\n{2}{Způsob šíření}

\n{1}{Psyb0t}
leden 2010 - mohl infikovat routery a vysokorychlostní modemy
\n{2}{Cíl útoku}
\n{2}{Předpokládaný počet napadených zařízení}
\n{2}{Identifikační znaky}
\n{2}{Způsob šíření}

\n{1}{Stuxnet}
17. červen 2010 - windows trojan, první worm attack na systémy SCADA (dohledová kontrola a sběr dat). Byl navržen aby cílil na Íránská jaderná zařízení. Používal platný certifikát Realtek.
\n{2}{Cíl útoku}
\n{2}{Předpokládaný počet napadených zařízení}
\n{2}{Identifikační znaky}
\n{2}{Způsob šíření}

\n{1}{Duqu}
1. září 2011 - předpokládá se, že je příbuzný stuxnetu. Název je z prefixu DQ souborů, které vytváří. CrySyS Lab napsal 60 stránkový elaborát.
\n{2}{Cíl útoku}
\n{2}{Předpokládaný počet napadených zařízení}
\n{2}{Identifikační znaky}
\n{2}{Způsob šíření}

\n{1}{NGRBot}
NGRBot září 2012 - červ využívá IRC network pro posílání souborů, posílání a příjem příkazů v mezi zombie sítí a útočníkovým IRC serverem. A monitoruje  a kontroluje síťovou konektivitu. Využívá techniku rootkitu v uživatelském módu ke skrývání a krádeži  informací o oběti.
\n{2}{Cíl útoku}
\n{2}{Předpokládaný počet napadených zařízení}
\n{2}{Identifikační znaky}
\n{2}{Způsob šíření}


% ============================================================================ %
% Pokud Vaše práce neobsahuje analytickou část, stačí odstranit či zakomentovat nasledujících pár rádků
%\cast{Analytická část}
%\n{1}{Nadpis}
%\n{2}{Podnadpis}
% ============================================================================ %
%\cast{Projektová část}
%\n{1}{Nadpis}
%\n{2}{Podnadpis}
% ============================================================================ %
\nn{Závěr}
\textcolor{red}{Na základě uvedených příkladů je zřejmé, že ransomware prošel od svého počátku velkým rozvojem. Z velké části se šíří pomocí emailových zpráv, kdy je přiložen samotný ransomware, nebo program, případně skript, který jej stáhne a naintaluje do uživatelova počítače. Úživatelé tak mohli zabránit infekci svého počítače pouze větší obezřejností. Na rozdíl od útoků, které zneužily tzv. zero-day zranitelnosti.}

\textcolor{red}{Dále po útoku ransomwaru \texttt{Reveton}, o kterém jsem psal v kapitole č.~\ref{reveton}, je znatelná změna ve vývoji ransomwaru, který používali sami vývojáři do modelu, kdy je ransomware poskytován za úplatu jako služba -- RaaS.}


% ============================================================================ %


\OdsazovaniOdstavcuStop


% ============================================================================ %
\seznamlit{
  % Na toto místo vložit veškeré citované bibliografické položky.
  % http://generator-citaci.cz/

	\bibitem{wiki-timeline}
	{
		Timeline of computer viruses and worms | Wikipedia.
		\it{Wikipedia, the free encyclopedia} [online]. [cit. 2022-11-18].
		Dostupné z: \url{https://en.wikipedia.org/wiki/Timeline_of_computer_viruses_and_worms}
	}
	
	\bibitem{zotob-schneider}
	{
		The Zotob Worm - Schneier on Security.
		\it{Schneier on Security} [online]. [cit. 2022-11-18].
		Dostupné z: \url{https://www.schneier.com/blog/archives/2005/11/the_zotob_worm.html}
	}
	
	\bibitem{zotob-wikidot}
	{
		Zotob - The Virus Encyclopedia.
		\it{The Virus Encyclopedia} [online]. [cit. 2022-11-18].
		Dostupné z: \url{http://virus.wikidot.com/zotob}
	}
	
	\bibitem{zotob-fsecure}
	{
		Zotob.A Description | F-Secure Labs.
		[online]. Copyright © F [cit. 18.11.2022].
		Dostupné z: \url{https://www.f-secure.com/v-descs/zotob_a.shtml}
	}
	\bibitem{zotobb-fsecure}
	{
		Zotob.B Description | F-Secure Labs.
		[online]. Copyright © F [cit. 18.11.2022].
		Dostupné z: \url{https://www.f-secure.com/v-descs/zotob_b.shtml}
	}
	\bibitem{zotobc-fsecure}
	{
		Zotob.C Description | F-Secure Labs.
		[online]. Copyright © F [cit. 18.11.2022].
		Dostupné z: \url{https://www.f-secure.com/v-descs/zotob_c.shtml}
	}
	
	\bibitem{zotob-ms}
	{
		Microsoft Security Advisory 899588 | Microsoft Learn.
		[online]. Copyright © Microsoft 2022 [cit. 18.11.2022]. 
		Dostupné z: \url{https://learn.microsoft.com/en-us/security-updates/securityadvisories/2005/899588}
	}
	
	\bibitem{zotob-msbulletin}
	{
		Microsoft Security Bulletin MS05-039 - Critical | Microsoft Learn
		[online]. Copyright © Microsoft 2022 [cit. 18.11.2022]. 
		Dostupné z: \url{https://learn.microsoft.com/en-us/security-updates/securitybulletins/2005/ms05-039}
	}
	
	\bibitem{zotob-msbulletin2}
	{
		Microsoft Security Bulletin MS04-007 - Critical | Microsoft Learn
		[online]. Copyright © Microsoft 2022 [cit. 18.11.2022]. 
		Dostupné z: \url{https://learn.microsoft.com/en-us/security-updates/securitybulletins/2004/ms04-007}
	}
	
	\bibitem{zotob-fbi}
	{
		FBI — Moroccan Authorities Sentence Two in Zotob Computer Worm Attack. 
		[online]. [cit. 2022-11-18].
		Dostupné z: \url{https://archives.fbi.gov/archives/news/pressrel/press-releases/moroccan-authorities-sentence-two-in-zotob-computer-worm-attack}
	}
	
	\bibitem{zotob-cnet}
	{
		Zotob damage deep but not widespread - CNET. 
		\it{CNET: Product reviews, advice, how-tos and the latest news}
		[online]. Copyright © 2022 CNET, a Red Ventures company. All rights reserved. [cit. 18.11.2022].
		Dostupné z: \url{https://www.cnet.com/news/privacy/zotob-damage-deep-but-not-widespread/}
	}
	
	\bibitem{nyxem-wiki}
	{
		Blackworm | Wikipedia.
		\it{Wikipedia, the free encyclopedia} [online]. [cit. 2022-11-21].
		Dostupné z: \url{https://en.wikipedia.org/wiki/Blackworm}
	}
	
	\bibitem{nyxem-caida}
	{
		David Moore, Colleen Shannon,
		The Nyxem Email Virus: Analysis and Inferences - CAIDA.
		\it{CAIDA} [online]. [cit. 2022-11-21].
		Dostupné z: \url{https://www.caida.org/archive/blackworm/}
	}
	\bibitem{nyxem-securelist}
	{
		Nyxem.e’s dreaded 32 bytes | Securelist.
		\it{Securelist | Kaspersky’s threat research and reports} [online].
		Copyright © 2022 AO Kaspersky Lab. All Rights Reserved. [cit. 21.11.2022].
		Dostupné z: \url{https://securelist.com/nyxem-es-dreaded-32-bytes/30127/}
	}
	\bibitem{nyxem-fsecure}
	{
		Email-Worm:W32/Nyxem.E Description | F-Secure Labs.
		[online]. Copyright © F [cit. 21.11.2022].
		Dostupné z: \url{https://www.f-secure.com/v-descs/nyxem_e.shtml}
	}
	\bibitem{nyxem-trendmicro}
	{
		WORM\_NYXEM.E - Threat Encyclopedia.
		[online]. Copyright © 2022 Trend Micro Incorporated. All rights reserved.
		[cit. 21.11.2022].
		Dostupné z: \url{https://www.trendmicro.com/vinfo/us/threat-encyclopedia/malware/WORM_NYXEM.E/}
	}
	\bibitem{nyxem-youtube}
	{
		Email-Worm.Win32.Nyxem.E - YouTube.
		\it{YouTube} [online]. Copyright © 2022 Google LLC [cit. 21.11.2022].
		Dostupné z: \url{https://www.youtube.com/watch?v=Fh0KxSuA0kY}
	}
}

% Pro generování literatury lze alternativně použít i příkaz "\seznamlitbib", 
% který se postará o plnohodnotné vkládání referencí pomocí "bibliography". 
% V takovém případě se využívají bibliografické údaje uložené v souboru 
% tex-literatura.bib. Ty se automaticky upravuji dle zvolené citační normy 
% (v šabloně je nastavena korektní česká norma).
%\seznamlitbib


% ============================================================================ %
% ============================================================================ %
% Encoding: UTF-8 (žluťoučký kůň úpěl ďábelšké ódy)
% ============================================================================ %

\seznamzkr

\begin{tabular}{ll}
	SMB		& Server Message Block\\	
	FTP		& File Transfer Protocol\\
	ASCII	& American Standard Code for Information Interchange\\
	SMTP		& Simple Mail Transfer Protocol\\

\end{tabular}

% ============================================================================ %
 % Seznam zkratek


% ============================================================================ %
\seznamobr  % Seznam je generován automaticky


% ============================================================================ %
\seznamtab  % Seznam je generován automaticky


% ============================================================================ %
%\input{tex/prilohy.tex} % Prilohy


% ============================================================================ %

\end{document}

% ============================================================================ %
