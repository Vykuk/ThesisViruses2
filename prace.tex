% ============================================================================ %
%
%           Šablona bakalářské/diplomové práce
%
% Autor:    Ing. Jozef Říha (4. květen 2006)
%           (některé komentáře převzaty z dokumentu Ivana Pomykacze)
%
% Verze:    2017-01-19, Ing. Pavel Tomášek (tomasek@fai.utb.cz)
%
% Kódování: UTF-8 (žluťoučký kůň úpěl ďábelšké ódy)
%
% Sazba:    pdflatex prace.tex && pdflatex prace.tex
%           (nutné dvakrát pro korektní vložení citací a jiných referencí),
%           v případě umístění literatury do externího bib souboru je třeba volat
%           pdflatex statement.tex && bibtex statement.tex && pdflatex statement.tex && pdflatex statement.tex
%
% Tip:      Ve správně vysázeném českém textu by na konci řádku neměla zůstant
%           samotná jednopísmenná předložka. Na takové místo se vkládá
%           nezalomitelná mezera pomocí symbolu ~. Existuje program, který umí
%           zpracovat celý TeX dokument najednou podle českých konvencí:
%           http://petr.olsak.net/ftp/olsak/vlna/
%
%           Pozor! Vzhledem k požadovanému standardu PDF/A nesmí obrázky obsahovat 
%           alfa kanál (průhlednost).
%
% ============================================================================ %


\documentclass[a4paper,12pt]{article}

% Definice vzhledu a nastavení se načítá z následujícího souboru (netřeba editovat)
\input{tex/UTB.tex}

% Uživatelské definice -- upravte dle požadavků
\nastavfakultu{FAI}
	% FAI  -- pro Fakultu aplikované informatiky
	% FAME -- pro Fakultu managementu a ekonomiky
	% FHS  -- pro Fakultu humanitních studií
	% FLKR -- pro Fakultu logistiky a krizového řízení
	% FMK  -- pro Fakutlu mutimediálních komunikací
	% FT   -- pro Fakultu technologickou
	% UNI  -- pro Univerzitní institut
\nastavtyp{SP}
	% BP   -- bakalářská práce
	% DP   -- diplomová práce
	% SP	   -- semestrální práce
\nastavrok{2022}
	% zadejte rok místo "xxxx"
\nastavjazyk{CZ}
	% CZ   -- práce bude v českém jazyce
	% EN   -- práce bude v anglickém jazyce

% Lze přidat vertikalni odsazeni nad (prvni parametr) a pod (druhy parametr)
% obrázky, tabulky i rovnice/soustavy rovnic
\nastavmezerukolemobrazku{0mm}{0mm}
\nastavmezerukolemtabulek{0mm}{0mm}
\nastavmezerukolemrovnic{0mm}{0mm}

\nastavautora{Bc. Petr Vykoukal}
\nastavnazevcz{Vývoj počítačových červů 2005-2022}
\nastavnazeven{Název práce anglicky (max. 2 řádky)} % Jen u anglicky psané práce
\nastavabstraktcz{Semestrální práce se zabývá vývojem počítačových od roku 2005 do současnosti a dále se věnuje několika vybraným červům, které se v tomto období vyskytli.}
\nastavabstrakten{The semester's thesis deals with the development of Computer Worms from 2005 to the present day and also focuses on several selected Worms that occurred during this period.}
\nastavklicovaslovacz{Malware, Worm, červ}
\nastavklicovaslovaen{Malware, Worm}

% Následující příkaz nastaví standard PDF/A-1b
\aplikujpdfa


%\usepackage{graphicx}
\usepackage{adjustbox}
\graphicspath{{./graphics/}}

\usepackage{booktabs}
\usepackage{array}

\usepackage{xurl}


%\hyphenation{u-kon-čo-val} - přidá do slovníku slovo ukončoval s možnostmi, jak ho rozdělovat. Odtud už stačí jen kdykoliv použít slovo "ukončoval"
%dalši varianta je použít slovo u\-kon\-čo\-val
\hyphenation{u-kon-čo-val na-pa-dal na-pa-de-né sy-sté-mem o-prá-vně-ní ran-som-ware
	to-má-še sou-bo-ry u-pra-vil od-stra-nit i-den-ti-fi-ko-vat bli-ka-jí-cím 
	in-sta-lo-vat způ-so-be-né al-go-rit-mem nej-čas-tě-ji ne-za-pla-ti-la
	ši-fro-va-cí pos-ky-tnu-té-ho při-chá-ze-ly žád-ná va-ri-an-ty web-ka-me-ra
	zra-ni-tel-nos-ti za-ši-fro-vá-ní sou-bo-ru vy-ge-ne-ro-val za-hr-nu-jí-cí
	na-pa-de-né-mu sys-té-mu in-fi-ko-va-né-mu pro-střed-nic-tvím u-ži-va-tel-ský-mi
	po-čí-ta-če e-mai-lo-vý-mi do-mov-ské}  

% ============================================================================ %
\begin{document}

\voffset=\valueVOffset\evensidemargin=\valueSideMargin\oddsidemargin=\valueSideMargin\headsep=\valueHeadSep\headheight=\valueHeadHeight\setlength{\parskip}{3pt}\textheight=\valueTextHeight\textwidth=\valueTextWidth

\titulnistrana
\clearpage

%\zadani
%následující řádek je místo zadání (nascanováná strana blablabla...)
\voffset=\valueVOffset\evensidemargin=\valueSideMargin\oddsidemargin=\valueSideMargin\headsep=\valueHeadSep\headheight=\valueHeadHeight\setlength{\parskip}{3pt}\textheight=\valueTextHeight\textwidth=\valueTextWidth

%\prohlaseni

\abstraktaklicovaslova


% ============================================================================ %
\clearpage
\thispagestyle{empty}
%Zde je místo pro případné poděkování, motto, úryvky knih, básní atp.


% ============================================================================ %
\obsah  % Obsah je generován automaticky


% ============================================================================ %
\OdsazovaniOdstavcuStart % Nastaví odsazování odstavců dle zvoleného jazyka

% ============================================================================ %
% Encoding: UTF-8 (žluťoučký kůň úpěl ďábelšké ódy)
% ============================================================================ %

% ============================================================================ %
\nn{Úvod}
V této práci se zabývám vývojem počítačovým červů v období od roku 2005 do současnosti. Vybral jsem 5 červů\cite{wiki-timeline}, o kterých pojednávám v následujících kapitolách. Červ je škodlivý software, který je specifický tím, že se dokáže sám šířit. Ke svému šíření používají různé techniky. Od rozesílání elektronickou poštou, nebo kopírování svého těla na přenosné, nebo sdílené jednotky až po zneužití zero-day zranitelností, kdy se dokáže sám instalovat do napadeného počítače bez nutnosti spolupráce uživatele.

% ============================================================================ %
%\cast{Teoretická část}

\n{1}{Zotob}
První mnou vybraný červ jménem \texttt{Zotob} se objevil 14. srpna 2005\cite{wiki-timeline}\cite{zotob-schneider}. Cílem tohoto červa byly počítače s operačním systémem Windows 2000\cite{zotob-fsecure}.

Po spuštění červa v napadeném počítači uložil svou kopii s názvem \texttt{botzor.exe} do systémové složky\cite{zotob-fsecure}. Dále vytvořil záznam v registrech systému, aby umožnil spuštění programu po startu operačního systému. Dále vytvořil mutex s názvem \texttt{B-O-T-Z-O-R}, aby zabráníl spuštění spuštění více, než jedné instance programu. V dalších verzích tohoto červa byl postup stejný, jen se soubory jmenovaly \texttt{csm.exe} ve verzi \texttt{Zotob.B} a~\texttt{per.exe} ve verzi \texttt{Zotob.C}.

Název použitého mutexu odpovídá názvu, kterým červa pojmenoval jeho autor. Tedy \texttt{Batzor2005}. Název \texttt{Zotob} pochází od antivirových společností, které hrozbu detekovaly\cite{zotob-wikidot}. Současně se skutečný název červa ve verzi \texttt{Zotob.A} a \texttt{Zotob.B} zapsán do souboru \texttt{hosts}. Konkrétně je tam tedy vepsána zpráva \uv{Botzor2005 Made By .... Greetz to good friend Coder. Based On HellBot3} a \uv{MSG to avs: the first av who detect this worm will be the first killed in the next 24hours!!!}\cite{zotob-fsecure}. Soubor \texttt{hosts} je také upraven, aby přesměroval přístup k některým serverům\footnote{Kompletní seznam je k vidění na stránce \url{https://www.f-secure.com/v-descs/zotob_a.shtml}} na adresu \texttt{127.0.0.1}, tedy \texttt{localhost}\cite{zotob-fsecure}. Také došlo ke spuštění FTP serveru na portu 33333, odkud si mohly další napadené systémy stáhnout tohoto červa\cite{zotob-fsecure}.

Ke svému šíření využíval tento červ ve všech verzích zranitelnost MS Windows Plug and Play služby označenou jako \texttt{MS05-039}\cite{zotob-fsecure}. Ve verzi \texttt{Zotob.A} vytvořil 300 vláken, ve verzcíh \texttt{Zotob.B} a \texttt{Zotob.C} pak 200 vláken, která zkoušely vytvořit spojení k náhodným IP adresám generovaným z IP adresy napadeného počítače podle masky podsítě 255.255.0.0\cite{zotob-fsecure}\cite{zotobb-fsecure}\cite{zotobc-fsecure}. Na nalezených IP adresách se pokusil navázal TCP spojení na portu 445 a využít výše uvedené zranitelnosti.

Jedná se ozranitelnost, která umožňuje vzálené spuštění kódu. Touto zranitelností trpělo více operačních systémů, ale právě u Windows 2000 bylo možné vzdálené spuštění kódu i bez ověření uživatele, zatím co na jiných systémech bylo nutné přihlášení\cite{zotob-ms}\cite{zotob-msbulletin}.

Pokud bylo úspěšně navázáno spojení s cílovým systémem protokolem TCP na portu 445, kde běží služba \texttt{Microsoft-ds} \texttt{SMB}\footnote{SMB -- Server Message Block} a podařilo se vzdálené spuštění kódu, došlo ke spuštění shellu na cílovém počítači, který naslouchal na portu 8888. Na tento port byl zaslán FTP sctipt, který zařídil stažení červa skrz FTP do napadeného počítače, kde běží FTP server na portu 33333, jako souboru \texttt{haha.exe} a jeho spuštění\cite{zotob-fsecure}\cite{zotobb-fsecure}.

Každý napadený počítač dokázal přijímat příkazy pomocí IRC a předdefinovaných adresách. Útočník tak mohl dát příkazy například k vyžádání informací o systému, stažení spustitelného souboru, manipulaci s bezpečnostním nastavením systému, nebo odstranení červa ze systému.

Verze \texttt{Zotob.C} pak dokázala využít zranitelnosti v Abstract Syntax Notation 1 popsané jako \texttt{MS04-007}, která učinila zranitelným i systém Windows XP do SP1\cite{zotob-msbulletin2}. Také tato verze využívala ke svému šíření metody phishingu pomocí spamování emailovými zprávami.\cite{zotobc-fsecure}

25. srpna 2005 byli zatčeni tři lidé zodpovědní za vznik tohoto červa. Jeho autorem byl osumnáctiletý Maročan, narozený v Rusku, Farid Essebar, který vystupoval pod přezdívkou \uv{Diab10} a jeho dvacetiletý přítel Achraf Bahloul. V Turecku byl zatčen také 21 letý Artilla Ekici s přezdívkou \uv{Coder}, který Essebarovi za vytvoření červa zaplatil.\cite{zotob-schneider}\cite{zotob-fbi}

Zasaženo bylo více než 100 amerických spočeností\cite{zotob-fbi}. Průměrná cena za zotavení z infekce červem \texttt{Zotob} je podle průzkumu společnosti Cybertrust 97\,000\,USD. 61 procent obětí uvedlo, že vyčištění vyžadovalo 80 hodin práce.\cite{zotob-wikidot}\cite{zotob-cnet} Původní zpráva společnosti však již není dostupná.

\n{1}{Nyxem.E}
Další z vybraných červů se objevil 20. ledna 2006\cite{nyxem-wiki}. Jeho cílem byl operační systém Windows. Je také znám pod dalšími jmény, jako \texttt{Blackworm}, \texttt{Kama Sutra}, \texttt{MyWife}, \texttt{CME-24}\dots\cite{nyxem-caida} Je napsán ve Visual Basicu a skompilován jako spustitelný soubor p-code a jeho velikost je 95kB. Jeho tělo obsahuje řetězce \uv{mysoulmystfly}, \uv{offlinehacker}, uv{evilpain} a \uv{setthesun}.\cite{nyxem-fsecure} 

Dokázal se šířit elektronickou poštou a nebo pomocí sdílených jednotek v síti. Napadenému uživateli tak mohl přijít jako příloha v emailu, kdy se snažil tvářit jako archiv \texttt{ZIP}. Pokud uživatel tento soubor spustil, otevřel se uživateli program WinZip, aby nabyl dojmu, že otevřel obyčejný archiv\cite{nyxem-trendmicro}. Současně také mohlo dojít k zablokování vstupů z klávesnice a myši, čím mělo být zabráněno uživateli v odhlášení, nebo spuštění správce úloh\cite{nyxem-fsecure}\cite{nyxem-trendmicro}. Také je kontaktována určitá webová stránka, která slouží jako počítadlo populace infikovaných počítačů.

V této chvíli červ vytvoří své kopie ve složce \texttt{\%WINDOWS\%}\footnote{Obvykle se 
jedná o složku C:\textbackslash Windows\textbackslash } soubor \texttt{rundll16.exe}, ve složce \texttt{\%SYSTEM\%}\footnote{Obvykle se jedná o složku C:\textbackslash Windows\textbackslash System32\textbackslash } soubory \texttt{scanregw.exe}, \texttt{Update.exe}, \texttt{Winzip.exe}, \texttt{WINZIP\_TMP.exe} a ve složce \texttt{\%SYSTEM ROOT\%\textbackslash Documents and Settings\textbackslash All Users\textbackslash Start Menu\textbackslash Pro\-grams\textbackslash Startup\textbackslash }\footnote{\%SYSTEEM ROOT\% je nahrazen kořenovým adresářem, kde je instalován Windows, tedy obvykle C:\textbackslash } soubor \texttt{Quick Pick.exe}.\cite{nyxem-fsecure}\cite{nyxem-trendmicro}

Následně červ upraví záznamy v registrech systému, aby v průzkumníku souborů zakázal zobrazení skrytých souborů, zajistil si spuštění při startu počítače a zabránil při startu spuštění různých antivirových produktů. Dále prohledá složky, kde se obvykle nachází antivirové programy, a pokusí se vymazat některé, případně všechny, jejich soubory. Mimo to se snaží odstanit i jiné programy, jako například \texttt{LimeWire}\footnote{LimeWire - je program určený pro P2P sdílení}.\cite{nyxem-caida}\cite{nyxem-fsecure}\cite{nyxem-trendmicro}

Ve chvíli, kdy je počítač infikován, tak kontroluje spouštěné programy a pokud nalezne titulek obsahující konkrétní řetězec, tak se pokusí program ukončit\cite{nyxem-fsecure}. Přiklad je k vidění i na YouTube\cite{nyxem-youtube}, kde v jedné části dochází k ukončení aplikace \texttt{Notepad}, které otevírá soubor \texttt{SCAN.txt}, čímž se jí do titulku dostalo hledané klíčové slovo.

Jednou měsíčně, konkrétně každého 3. dne v měsíci, počínaje dnem 3. 2. 2006, je pak spuštěn program \texttt{Update.exe}, který v počítači vyhledá všechny soubory s koncovkami \texttt{DOC}, \texttt{XLS}, \texttt{MDB}, \texttt{MDE}, \texttt{PPT}, \texttt{PPS}, \texttt{ZIP}, \texttt{RAR}, \texttt{PDF}, \texttt{PSD} a \texttt{DMP} a obsah těchto souborů přešpíše 32B dlouhým řetězcem, který převeden do ASCII zní \uv{DATA ERROR [47 0F 94 93 F4 K5]}. Poslední 2B jsou hodnoty \texttt{0D 0A}, což jsou netisknutelné znaky pro konec řádku.\cite{nyxem-securelist}\cite{nyxem-fsecure}

V počítači a ve složce, kde je uložena cache programu Internet Explorer vyhledává soubory s určitými koncovkami a snaží se z nich vytěžit emailové adresy. Pokud tyto adresy ale obsahují určitá klíčová slova, jako například \uv{mcafee}, nebo \uv{spam}, tak je ignoruje\cite{nyxem-fsecure}. Červ obsahuje svůj vlastní SMTP engine\cite{nyxem-trendmicro} a pomocí něj se snaží šířit na vytěžené emailové adresy. Má své předdefinované předměty a těla zpráv. Sám sebe pak do zpráv připojí jako přílohu.\cite{nyxem-fsecure}

Pro šíření v síti má dvě rutiny. První z nich prohlídne v registru umístění složek, kde jsou osobní dokumenty, nebo nedávno otevřené dokumenty. Pokud takové nalezne, tak složky projde a náhodně si vybere název souboru a připojí k němu koncovku \texttt{EXE}. Následně vyhledá sdílené jednotky v síti a na vyhledané jednotce vytvoří nový soubor se vzniklým jménem, který ale obsahuje tělo červa. Pokud nenalezne žádný vhondý soubor, jehož název by mohl okopírovat, tak použije \uv{New WinZip File.exe}, \uv{Zipped Files.exe}, nebo \uv{movies.exe}.\cite{nyxem-fsecure}

Druhá varianta je pak, že prohledá sdílené jednotky v síti a tam se snaží zkopírovat jako:
\begin{itemize}
	\item \textbackslash Admin\$\textbackslash WINZIP\_TMP.exe
	\item \textbackslash c\$\textbackslash WINZIP\_TMP.exe
	\item \textbackslash c\$\textbackslash Documents and Settings\textbackslash All Users\textbackslash Start Menu\textbackslash Programs\textbackslash Startup\textbackslash WinZip Quick Pick.exe
	\begin{itemize}
		\item V tomto případě ve stejné složce odstraní soubor \uv{WinZip Quick Pick.lnk}
	\end{itemize}
\end{itemize}
Také se snaží zjistit, zda existuje na sdílené jednotce antivirový program tím, že prohledává konkrétná složky jako napříkald \uv{\textbackslash c\$\textbackslash Program Files\textbackslash Norton AntiVirus}, a pokud nalezne, tak se je snaží odstanit. Také se pokusí na vzáleném počítači spustit plánovanou úlohu na 59. minutu aktuální hodiny, která by spustila nakopírované tělo červa.\cite{nyxem-fsecure}

Tím, že \texttt{Nyxem} při své instalaci kontaktuje určitou webovou stránku, která funguje jako počítadlo populace, je mělo by být možné odhadnout kolik počítačů bylo infikováno. Dle analýzy Davida Moora a Collen Shannon je odhad infikovaných počítačů mezi 469\,507 až 946\,835 ve více než 200 zemích za obcodbí 15. ledna 2006 23:40:45\,UTC do 1. února 2006 05:00:12\,UTC.\cite{nyxem-caida}

Přes to však dle zprávy společnosti F-Secure může být započítáno více přístupů z jedné IP adresy. Výsledné číslo podle nich pak vychází na více než 300\,000 unikátních IP adres.\cite{nyxem-fsecure}

\n{1}{Conficker}
V listopadu 2008 byl poprvé detekován červ, který byl pojmenován \texttt{Conficker}, nebo také \texttt{Downup}, \texttt{Downadup}, nebo \texttt{Kido}. Tento červ cílil na zranitelnost \texttt{MS08-067}, kterou trpěly počítače s operačním systémem Windows.\cite{conficker-wiki}

Zranitelnost umožňovala vzdálené spuštění kódu na počítačích s operačním systémem Windows 2000 (SP4), Windows XP (SP2 a SP3), Windows Server 2003 (SP1 a~SP2), Windows Vista, Windows Server 2008 a Windows 7 pomocí protokolu RPC\footnote{Remote Procedure Call -- klient-server služba, kdy klient může vyžádat službu od programu umístěného na jiném počítači}. Zneužitím této chyby je možné na vzdáleném počítači manipulovat s daty, instalovat aplikace, nebo vytvořit uživatelský účet s právy administrátora.\cite{conficker-ms}\cite{conficker-malwarebytes}

Na operačních systémech Windows 2000, Windows XP a Windows Server 2003 je možné této zranitelnosti využít anonymním uživatelem, tedy kýmkoliv, kdo má přístup k dané síti. Na ostatních operačních systémech vyjmenovaných výše je však nutné ověření uživatele. Operační systému Windows 7 byl zranitelný jen ve verzi \uv{Pre-Beta}.\cite{conficker-ms}

Pomocí výše uvedené zranitelnosti operačních systémů Windows se šířil dokázal šířit téměř ve všech svých verzích. Ve verzích \texttt{Conficker.B} a \texttt{Conficker.C} se pak dokázal šířit prostřednictvým vyjímatelných médií, jako jsou USB Flash, nebo na sdíléné diskové jednotky ostatních počítačů v síti, kam se zkoušel připojit pomocí slovníkového útoku přes účet Administrátora.\cite{conficker-wiki}\cite{conficker-trendmicro}\cite{conficker-fsecure}

K šíření pomocí vyjímatelných médií využíval výchozího nastavení operačního systému, kdy bylo povoleno automatické spuštění. Při připojení vyjímatelného disku k infikovanému počítači tak byl vytvořen soubor \texttt{autorun.inf}, který po připojení k novému počítači s výchozím nastavením zajistil spuštění těla červa. Samotné tělo pak uložil do složky \texttt{RECYCLER} s náhodným názvem souboru a koncovkou.\cite{conficker-trendmicro}

Při spuštění na čistém počítači uložil své tělo jako knihovnu \texttt{DLL} s náhodným názvem do složky \texttt{\%SYSTEM\%}\footnote{Obvykle se jedná o složku C:\textbackslash Windows\textbackslash System32\textbackslash}, případně do složky programu \texttt{Internet Explorer}, nebo \texttt{Movie Maker}.\cite{conficker-trendmicro}

Následně pak upravil registry systému, aby si zajistil spuštění po startu prostřednictvím legitimního souboru \texttt{rundll32.exe}, nebo \texttt{svchost32.exe}. Aby zamezil svému nalezení, upravil také registry, aby se v průzkumníku souborů nezobrazovaly skryté a~systémové soubory.\cite{conficker-trendmicro}

V počítači zakázal běh některých služeb, zejména Windows Automatic Update Service, nebo Windows Defender Service a pokud zjistil, že běží na operačním systému Windows Vista, pak zakázal automatické ladění protokolu TCP/IP. Napadne API operačního systému, poskytující komunikaci s DNS, aby zabránil přístupu k určitým doménám, které jsou zaměřeny na počítačovou bezpečnost.\cite{conficker-fsecure}

Každý den červ kontaktoval jednu z pěti domén, aby získal aktuální datum, pomocí kterého generoval seznam domén, odkud mohl stáhnout další soubory.\cite{conficker-fsecure}

Odhad počtu napadených počítaču ze dne 16. ledna 2009 dle společnosti F-Secure je necelých 9 miliónů\cite{conficker-fsecure2}. Podle článku na serveru ZDNET z 10. června 2015 mohlo být stále ještě infikováno okolo 500\,000 zařízení. Společnost Microsoft dokonce vypsala odměnu 250\,000\,USD za informace vedoucí k zatčení osob, zodpovědných za vypuštění tohoto červa\cite{conficker-zdnet}.

\n{1}{Stuxnet}
15. června 2010 byl identifikován červ \texttt{Stuxnet}, který byl vyvíjen již od roku 2005. V lednu 2010 bylo prověřováno, proč v závodu na obohacování uranu v Natanzu selhávají ve větší míře centrifugy na obohacování uranu. Až po dalších pěti měsících se výzkumníkům podařilo nalézt škodivé soubory v jednom systému. Červ dokázal infikovat více než 20\,000 zařízení ve 14 Íránskách jaderných zařízeních a zničil okolo 900 centrifug.\cite{stuxnet-nordvpn}

Vzhledem k tomu se \texttt{Stuxnet} považuje za kybernetickou zbraň, vytvořenou právě k tomu, aby napadl právě Íranská jaderná zařízení a narušil tak Íránský jaderný program\cite{stuxnet-avast}. Vzhledem k tomu, že jaderná zařízení nejsou připojena k internetu, byl tento červ navržen tak, aby se dokázal efektivně šířit pomocí vyměnitelných médií\cite{stuxnet-nordvpn}.

Ka svému šíření využíval zranitelnost \texttt{CVE-2010-2568}, která spočívala v tom, jak operační systémy Windows zachází se soubory \texttt{LNK}, což je odkaz, sloužící ke spuštění programu. Tuto zranitelnost měly operační systémy Windows XP, Windows Server 2003, Windows Vista, Windows Server 2008 a Windows 7. U všech zmiňovaných operačních systému je zranitelnost označena jako kritická. Speciálně upravený \texttt{LNK} soubor měl nastavenou ikonu, ale při pokusu o zobrazení této ikony v průzkumníku souboru, nebo prohlížeči Internet Explorer, došlo ke spuštění škodlivého kódu s uživatelskými právy právě přihlášeného uživatele.\cite{stuxnet-ms}

Kromě těchto operačních systému, se červ zaměřoval také na řídící systém Siemens PCS 7, WinCC a STEP7, což je software, který slouží k řízení, sběru a reprezentaci dat z řídících systémů\cite{stuxnet-nordvpn}.

Při spuštění byly na napadeném počítači vytvořeny dva soubory \texttt{mrxcls.sys} a \texttt{mrxnet.sys}, které se uložily do složky \texttt{C:\textbackslash System32\textbackslash Drivers}. Do další složky \texttt{C:\textbackslash Win\-dows\textbackslash inf} byly pak uloženy další důležité \texttt{PNF} soubory, z nichž hlavní komponenta je \texttt{oem7a.PNF}, což je šifrovaný soubor \texttt{DLL}.\cite{stuxnet-fsecure}

Po úspěšném spuštění a detekci, za jsou k počítači připojeny systémy pro řídící jednotky Siemens jsou upravovány signály senzorů, které v řídícím systému mohou bezpečně vypnout řízený proces. Také upravil program v řídících jednotkách, aby ovlivnil fkrekvenční měniče pohonů centrifug. Ty pak točily dlouhou dobu vysokou rychlostí, aby došlo k jejich poškození. Údaje ze senzorů ale byly falšovány a obsluha nemohla zjistit, že něco není v pořádku.\cite{stuxnet-nordvpn}\cite{stuxnet-avast}

\texttt{Stuxnet} se ale dostal i na počítače, které byly připojeny k Internetu a mohl se tak rozšířit i mimo cílová zařízení a mohl infikovat více, než 200\,000 zařízení\cite{stuxnet-zdnet}. Jeho kód tam mohl sloužit jako inspirace pro vznit dalších červů\cite{stuxnet-avast}.

\n{1}{Koobface}
Jako poslední jsem si vybral červ \texttt{Koobface}. A to z důvodu, že dokázal cílit nejen na operační systémy Windows, ale také Mac OS a Linux. Poprvě byl objeven v roce 2008 a byl velmi aktivní v roce 2009. Po následném poklesu jeho aktivity se však opět objevil v roce 2013.\cite{koobface-kaspersky}

K tomu, aby se šířil, tak využíval technik sociálního inženýrštví a phishingu. Nejčastěji se jednalo o odkaz na sociální síti ke zhlédnutí videa. Po klikntí na odkaz byl uživatel přesměrován na podrvženou stráku připomínající YouTube, nebo Facebook. Uživateli ale byl vnucen update Flash přehrávače\cite{koobface-naked}. Případně byl spuštěn Java applet, který vyzíval uživatele k souhlasu. Applet se pokusil využít zranitelnost neaktualizované verze Java, zjistil verzi a typ operačního systému uživatele a na základě toho instaloval červa do počítače bez nutnosti souhlasu uživatele\cite{koobface-softpedia}.

Po spuštění se červ nainstaloval, v případě operačního systému Windows, do složky s operačním systémem pod názvem \uv{freddy35.exe}\cite{koobface-fsecure}. Jsou ale známy i další názvy souborů, pod kterými se dokáže uložit\cite{koobface-kaspersky}. Následně upraví registry systému, aby došlo ke spouštění tohoto souboru při každém startu operačního systému. Provede také změnu v registrech, aby docházelo automaticky ke zobrazení MIME\footnote{MIME je rozšíření původní elektronické pošty, která podporovala přenos pouze ASCII znaků, o~další možnosti}\cite{koobface-fsecure}.

Při napadení a instalaci červa do operačního systému Mac OS se červ usídlil v~domovské složce uživatele ve skryté složce \texttt{.jnana}.

U operačního systému Linux byly jeho možnosti značně omezené. Veškeré soubory si, jak v případě Mac OS, uložil v domovské složce uživatele do složky s názvem \texttt{.jnana}, ale program byl schopen běžet pouze do restartu systému. Tím je infekce operačního systému Linux na rozdíl od operačního systému Windows a Mac OS pouze dočasná.\cite{koobface-linux}\cite{koobface-softpedia}

Zdá se tedy, že platformy, na které červ přímo cílil, jsou Windows a Mac OS. To, že červ dokázal napadnout i operační systém Linux se zdá být spíše vedlejším efekte, než záměrem. Už proto, že infekce počítače s operačním systémem Linux je jen dočasná, tedy pouze není dořešeno automatické spuštění při startu systému.\cite{koobface-softpedia2}

Po úspěšné instalaci červa do systému se tento systém zařadil do sítě botnetů. Byl schopen přijímat příkazy z C\&C serveru a stahovat a instalovat do systému další škodlivý software. Červ se také snaží z napadeného počítače vytěžit přihlažovací údaje do sociálních sítí a vyhledávat kontakty, kam se dále pokusí pomocí sociálních sítí šířit.\cite{koobface-fsecure}

Předpokládá se, že červ mohl infikovat až 800\,000 zařízení.\cite{koobface-bbc}

% ============================================================================ %
% Pokud Vaše práce neobsahuje analytickou část, stačí odstranit či zakomentovat nasledujících pár rádků
%\cast{Analytická část}
%\n{1}{Nadpis}
%\n{2}{Podnadpis}
% ============================================================================ %
%\cast{Projektová část}
%\n{1}{Nadpis}
%\n{2}{Podnadpis}
% ============================================================================ %
\nn{Závěr}
Z příkladů uvedených v této práci vyplívá, že některé počítačové červy mohou spoléhat na metody sociálního inženýrství a uživatele napadeného počítače nějakou manipulací přimět k instalaci škodlivého software, ale ty sofistikovanější dokáží napadat počítače zcela sami tím, že zneužijí tzv. \uv{zero-day} zranitelnosti systému, případně spoléhají na to, že se rozšíří na systémy, kde nejsou instalovány bezpečnostní záplaty. Využívány jsou i techniky, při kterých jsou využívána slabá hesla uživatelů, nebo administrátorů na napadaných systémech a ta jsou prolomena pomocí slovníkového útoku.


% ============================================================================ %


\OdsazovaniOdstavcuStop


% ============================================================================ %
\seznamlit{
  % Na toto místo vložit veškeré citované bibliografické položky.
  % http://generator-citaci.cz/

	\bibitem{wiki-timeline}
	{
		Timeline of computer viruses and worms | Wikipedia.
		\it{Wikipedia, the free encyclopedia} [online]. [cit. 2022-11-18].
		Dostupné z: \url{https://en.wikipedia.org/wiki/Timeline_of_computer_viruses_and_worms}
	}
	
	\bibitem{zotob-schneider}
	{
		The Zotob Worm - Schneier on Security.
		\it{Schneier on Security} [online]. [cit. 2022-11-18].
		Dostupné z: \url{https://www.schneier.com/blog/archives/2005/11/the_zotob_worm.html}
	}
	
	\bibitem{zotob-wikidot}
	{
		Zotob - The Virus Encyclopedia.
		\it{The Virus Encyclopedia} [online]. [cit. 2022-11-18].
		Dostupné z: \url{http://virus.wikidot.com/zotob}
	}
	
	\bibitem{zotob-fsecure}
	{
		Zotob.A Description | F-Secure Labs.
		[online]. Copyright © F [cit. 18.11.2022].
		Dostupné z: \url{https://www.f-secure.com/v-descs/zotob_a.shtml}
	}
	\bibitem{zotobb-fsecure}
	{
		Zotob.B Description | F-Secure Labs.
		[online]. Copyright © F [cit. 18.11.2022].
		Dostupné z: \url{https://www.f-secure.com/v-descs/zotob_b.shtml}
	}
	\bibitem{zotobc-fsecure}
	{
		Zotob.C Description | F-Secure Labs.
		[online]. Copyright © F [cit. 18.11.2022].
		Dostupné z: \url{https://www.f-secure.com/v-descs/zotob_c.shtml}
	}
	
	\bibitem{zotob-ms}
	{
		Microsoft Security Advisory 899588 | Microsoft Learn.
		[online]. Copyright © Microsoft 2022 [cit. 18.11.2022]. 
		Dostupné z: \url{https://learn.microsoft.com/en-us/security-updates/securityadvisories/2005/899588}
	}
	
	\bibitem{zotob-msbulletin}
	{
		Microsoft Security Bulletin MS05-039 - Critical | Microsoft Learn
		[online]. Copyright © Microsoft 2022 [cit. 18.11.2022]. 
		Dostupné z: \url{https://learn.microsoft.com/en-us/security-updates/securitybulletins/2005/ms05-039}
	}
	
	\bibitem{zotob-msbulletin2}
	{
		Microsoft Security Bulletin MS04-007 - Critical | Microsoft Learn
		[online]. Copyright © Microsoft 2022 [cit. 18.11.2022]. 
		Dostupné z: \url{https://learn.microsoft.com/en-us/security-updates/securitybulletins/2004/ms04-007}
	}
	
	\bibitem{zotob-fbi}
	{
		FBI — Moroccan Authorities Sentence Two in Zotob Computer Worm Attack. 
		[online]. [cit. 2022-11-18].
		Dostupné z: \url{https://archives.fbi.gov/archives/news/pressrel/press-releases/moroccan-authorities-sentence-two-in-zotob-computer-worm-attack}
	}
	
	\bibitem{zotob-cnet}
	{
		Zotob damage deep but not widespread - CNET. 
		\it{CNET: Product reviews, advice, how-tos and the latest news}
		[online]. Copyright © 2022 CNET, a Red Ventures company. All rights reserved. [cit. 18.11.2022].
		Dostupné z: \url{https://www.cnet.com/news/privacy/zotob-damage-deep-but-not-widespread/}
	}
	
	\bibitem{nyxem-wiki}
	{
		Blackworm | Wikipedia.
		\it{Wikipedia, the free encyclopedia} [online]. [cit. 2022-11-21].
		Dostupné z: \url{https://en.wikipedia.org/wiki/Blackworm}
	}
	
	\bibitem{nyxem-caida}
	{
		David Moore, Colleen Shannon,
		The Nyxem Email Virus: Analysis and Inferences - CAIDA.
		\it{CAIDA} [online]. [cit. 2022-11-21].
		Dostupné z: \url{https://www.caida.org/archive/blackworm/}
	}
	\bibitem{nyxem-securelist}
	{
		Nyxem.e’s dreaded 32 bytes | Securelist.
		\it{Securelist | Kaspersky’s threat research and reports} [online].
		Copyright © 2022 AO Kaspersky Lab. All Rights Reserved. [cit. 21.11.2022].
		Dostupné z: \url{https://securelist.com/nyxem-es-dreaded-32-bytes/30127/}
	}
	\bibitem{nyxem-fsecure}
	{
		Email-Worm:W32/Nyxem.E Description | F-Secure Labs.
		[online]. Copyright © F [cit. 21.11.2022].
		Dostupné z: \url{https://www.f-secure.com/v-descs/nyxem_e.shtml}
	}
	\bibitem{nyxem-trendmicro}
	{
		WORM\_NYXEM.E - Threat Encyclopedia.
		[online]. Copyright © 2022 Trend Micro Incorporated. All rights reserved.
		[cit. 21.11.2022].
		Dostupné z: \url{https://www.trendmicro.com/vinfo/us/threat-encyclopedia/malware/WORM_NYXEM.E/}
	}
	\bibitem{nyxem-youtube}
	{
		Email-Worm.Win32.Nyxem.E - YouTube.
		\it{YouTube} [online]. Copyright © 2022 Google LLC [cit. 21.11.2022].
		Dostupné z: \url{https://www.youtube.com/watch?v=Fh0KxSuA0kY}
	}
	
	\bibitem{conficker-wiki}
	{
		Conficker | Wikipedia.
		\it{Wikipedia, the free encyclopedia} [online]. [cit. 2022-11-26].
		Dostupné z: \url{https://en.wikipedia.org/wiki/Conficker}
	}	
	\bibitem{conficker-ms}
	{
		Microsoft Security Bulletin MS08-067 - Critical | Microsoft Learn.
		[online]. Copyright © Microsoft 2022 [cit. 26.11.2022].
		Dostupné z: \url{https://learn.microsoft.com/en-us/security-updates/securitybulletins/2008/ms08-067}
	}
	\bibitem{conficker-malwarebytes}
	{
		Worm.Conficker | Malwarebytes Labs.
		\it{Malwarebytes Cybersecurity for Home and Business | Anti-Malware \& Antivirus}
		[online]. Copyright © 2022 All Rights Reserved [cit. 26.11.2022].
		Dostupné z: \url{https://www.malwarebytes.com/blog/detections/worm-conficker}
	}
	\bibitem{conficker-trendmicro}
	{
		CONFICKER - Threat Encyclopedia.
		[online]. Copyright © 2022 Trend Micro Incorporated. All rights reserved. [cit. 26.11.2022].
		Dostupné z: \url{https://www.trendmicro.com/vinfo/us/threat-encyclopedia/malware/conficker}
	}
	\bibitem{conficker-fsecure}
	{
		Worm:W32/Downadup.AL Description | F-Secure Labs.
		[online]. Copyright © F [cit. 26.11.2022].
		Dostupné z: \url{https://www.f-secure.com/v-descs/worm_w32_downadup_al.shtml}
	}
	\bibitem{conficker-fsecure2}
	{
		News from the Lab Archive : January 2004 to September 2015.
		\it{Archive | F-Secure} [online]. [cit. 26.11.2022].
		Dostupné z: \url{https://archive.f-secure.com/weblog/archives/00001584.html}
	}
	\bibitem{conficker-zdnet}
	{
		Opening up a can of worms: Why won't Conficker just die, die, die? | ZDNET.
		\it{News and Advice on the World's Latest Innovations | ZDNET}
		[online]. Copyright © 2022 ZDNET, A Red Ventures company. All rights reserved. [cit. 26.11.2022].
		Dostupné z: \url{https://www.zdnet.com/article/opening-up-a-can-of-worms-why-wont-conficker-just-die-die-die/}
	}
	
	\bibitem{stuxnet-nordvpn}
	{
		Stuxnet explained — the worm that went nuclear | NordVPN. 
		\it{The best online VPN service for speed and security | NordVPN}
		[online]. Copyright © 2022 Nord Security. All Rights Reserved [cit. 27.11.2022]. 
		Dostupné z: \url{https://nordvpn.com/blog/stuxnet-virus/}
	}
	\bibitem{stuxnet-avast}
	{
		What is Stuxnet, Who Created it \& How Does it Work? | Avast.
		[online]. [cit. 27.11.2022].
		Dostupné z: \url{https://www.avast.com/c-stuxnet}
	}
	\bibitem{stuxnet-ms}
	{
		Microsoft Security Bulletin MS10-046 - Critical | Microsoft Learn. 
		[online]. Copyright © Microsoft 2022 [cit. 27.11.2022].
		Dostupné z: \url{https://learn.microsoft.com/en-us/security-updates/securitybulletins/2010/ms10-046}
	}
	\bibitem{stuxnet-fsecure}
	{
		Trojan-Dropper:W32/Stuxnet Description | F-Secure Labs.
		[online]. Copyright © F [cit. 27.11.2022].
		Dostupné z: \url{https://www.f-secure.com/v-descs/trojan-dropper_w32_stuxnet.shtml}
	}
	\bibitem{stuxnet-zdnet}
	{
		The world's most famous and dangerous APT (state-developed) malware | ZDNET.
		\it{News and Advice on the World's Latest Innovations | ZDNET}
		[online]. Copyright © 2022 ZDNET, A Red Ventures company. All rights reserved. [cit. 27.11.2022].
		Dostupné z: \url{https://www.zdnet.com/pictures/the-worlds-most-famous-and-dangerous-apt-state-developed-malware/3/}
	}
	
	\bibitem{koobface-kaspersky}
	{
		What Is the Koobface Virus?
		\it{Kaspersky Cyber Security Solutions for Home and Business | Kaspersky}
		[online]. Copyright © [cit. 01.12.2022].
		Dostupné z: \url{https://www.kaspersky.com/resource-center/definitions/what-is-the-koobface-virus}
	}
	\bibitem{koobface-naked}
	{
		Q\&A about the Koobface virus – Naked Security.
		\it{Naked Security – Computer Security News, Advice and Research}
		[online]. Copyright © 1997 [cit. 01.12.2022].
		Dostupné z: \url{https://nakedsecurity.sophos.com/questions-and-answers-about-koobface/}
	}
	\bibitem{koobface-fsecure}
	{
		Net-Worm:W32/Koobface.ES Description | F-Secure Labs.
		[online]. Copyright © F [cit. 01.12.2022].
		Dostupné z: \url{https://www.f-secure.com/v-descs/net-worm_w32_koobface_es.shtml}
	}
	\bibitem{koobface-linux}
	{
		Cross-Platform Koobface Worm Can Infect Linux.
		\it{Tux Tweaks - Linux Tweaks, HowTo's and Reviews}
		[online]. [cit. 01.12.2022].
		Dostupné z: \url{https://tuxtweaks.com/2010/10/cross-platform-koobface-worm-can-infect-linux/}
	}
	\bibitem{koobface-softpedia}
	{
		New Koobface Variant Infects Linux Systems
		\it{www.softpedia.com} © 2001-2022 Softpedia
		[online]. [cit. 01.12.2022].
		Dostupné z: \url{https://news.softpedia.com/news/New-Koobface-Variant-Infects-Linux-too-163450.shtml}
	}
	\bibitem{koobface-mac}
	{
		Koobface Worm Targets Java on Mac OS X – Krebs on Security.
		\it{Krebs on Security – In-depth security news and investigation}
		[online]. Copyright © Krebs on Security [cit. 01.12.2022].
		Dostupné z: \url{https://krebsonsecurity.com/2010/10/koobface-worm-targets-java-on-mac-os-x/}
	}
	\bibitem{koobface-softpedia2}
	{
		Linux Java-Based Trojan Might Have Been an Accident
		\it{www.softpedia.com} © 2001-2022 Softpedia
		[online]. [cit. 03.12.2022].
		Dostupné z: \url{https://news.softpedia.com/news/Linux-Java-Based-Trojan-Might-Have-Been-an-Accident-163848.shtml}
	}
	\bibitem{koobface-bbc}
	{
		Facebook Koobface worm 'hacker gang named' - BBC News.
		\it{BBC - Homepage}
		[online]. Copyright © 2022 BBC. The BBC is not responsible for the content of external sites. [cit. 03.12.2022].
		Dostupné z: \url{https://www.bbc.com/news/technology-16595856}
	}
}

% Pro generování literatury lze alternativně použít i příkaz "\seznamlitbib", 
% který se postará o plnohodnotné vkládání referencí pomocí "bibliography". 
% V takovém případě se využívají bibliografické údaje uložené v souboru 
% tex-literatura.bib. Ty se automaticky upravuji dle zvolené citační normy 
% (v šabloně je nastavena korektní česká norma).
%\seznamlitbib


% ============================================================================ %
% ============================================================================ %
% Encoding: UTF-8 (žluťoučký kůň úpěl ďábelšké ódy)
% ============================================================================ %

\seznamzkr

\begin{tabular}{ll}
	SMB		& Server Message Block\\	
	FTP		& File Transfer Protocol\\
	ASCII	& American Standard Code for Information Interchange\\
	SMTP		& Simple Mail Transfer Protocol\\

\end{tabular}

% ============================================================================ %
 % Seznam zkratek


% ============================================================================ %
%\seznamobr  % Seznam je generován automaticky


% ============================================================================ %
%\seznamtab  % Seznam je generován automaticky


% ============================================================================ %
%\input{tex/prilohy.tex} % Prilohy


% ============================================================================ %

\end{document}

% ============================================================================ %
